%--------------------
% Packages
% -------------------
\documentclass[a4paper]{article}
\usepackage[utf8x]{inputenc}
\usepackage[T1]{fontenc}
\usepackage[pdftex]{graphicx} % Required for including pictures
\usepackage[pdftex,linkcolor=black,pdfborder={0 0 0}]{hyperref} % Format links for pdf
\usepackage{calc} % To reset the counter in the document after title page
\usepackage{enumitem} % Includes lists
\frenchspacing % No double spacing between sentences
\linespread{1.2} % Set linespace
\usepackage[a4paper, lmargin=0.1\paperwidth, rmargin=0.1\paperwidth, tmargin=0.1111\paperheight, bmargin=0.1111\paperheight]{geometry} %margins
\usepackage[all]{nowidow} % Tries to remove widows
\usepackage[protrusion=true,expansion=true]{microtype} % Improves typography, load after fontpackage is selected
\usepackage{pdfpages}
\usepackage[title]{appendix}
\usepackage{listings}
\usepackage{color}
\usepackage{multicol}
\usepackage{amsmath}
\usepackage{colonequals}
\usepackage{amssymb}
\usepackage{graphicx}
\usepackage{subfig}
\usepackage{float}
\usepackage{ mathrsfs }
\usepackage{svg}
\DeclareSymbolFont{symbolsC}{U}{pxsyc}{m}{n}
\DeclareMathSymbol{\coloneqq}{\mathrel}{symbolsC}{"42}
\usepackage{algorithm}
\usepackage[noend]{algpseudocode}
%-----------------------
% Title, Authors
%-----------------------
\title{Alternate Proof of: LR $ \sim \chi^2$}
%-----------------------
% Begin document
%-----------------------
\begin{document}
\maketitle
%-----------------------
% Begin main section
%-----------------------
In class we went over the fact that the log-likelihood ratio (LR) is asymptotically
$\chi^2$ distributed. I think you mentioned a complicated proof by Wilks (1938)
\cite{wilks38} that originally showed it. I just wanted to mention another proof of
it that maybe isn't totally airtight but I thought was interesting.

This property of the LR came up when we were talking about MLE (I think) and you
were comparing the likelihood of the unconstrained model $\mathcal{L}_u$ and the
constrained model $\mathcal{L}_c$. Specifically you noted that,
$$
LR =
$$



\newpage
%-----------------------
% Begin Bibliography
%-----------------------
\renewcommand{\refname}{References}
\begin{thebibliography}{9}
\bibitem{wilks38}
	Wilks, Samuel S. The large-sample distribution of the likelihood ratio for testing composite hypotheses. The Annals of Mathematical Statistics 9.1 (1938): 60-62.
\end{thebibliography}

\end{document}
